\section{Discusión}
Empezamos comparando las tablas \ref{tab:bruteAT} y \ref{tab:bruteLT}. En las mismas se observa que, como se esperaba, el número de iteraciones no se  ve alterado por el uso de ArrayList o LinkedList. También se observa con las gráficas \ref{fig:bruteAG} y \ref{fig:bruteLG} que la complejidad es la misma para ambas estructuras. Por otra parte, calculando el promedio de los tiempos de ambas usando Excel obtenemos un tiempo promedio de $4,01504*10^{11}$ ms usando LinkedList y de 32596966541 ms usando ArrayList, por lo tanto, la implementación de fuerza bruta con LinkedList consume más tiempo en promedio que la que usa ArrayList.\\

Comparando las tablas \ref{tab:divideAT} y \ref{tab:divideLT} se encuentra que el número de iteraciones ya no es igual para cada cantidad de puntos, siendo en general la implementación con LinkedList la que requiere más iteraciones para cada caso. Sin embargo, dado a que el número de iteraciones usando divide y vencerás es un promedio y la cantidad de iteraciones siempre puede variar para cada prueba, no se puede afirmar concluyentemente que el uso de LinkedList causo este aumento en las iteraciones promedio. Además, si se observan las gráficas \ref{fig:divideAG} y \ref{fig:divideLG} se nota que ambas siguen una tendencia lineal. Finalmente, calculando el promedio de los tiempos de ambas obtenemos que la implementación con LinkedList tarda en promedio 72476330,77 ms y la implementación con ArrayList tarda en promedio 14699517,65 ms, por lo tanto, la implementación de divide y vencerás con LinkedList consume más tiempo en promedio que la que usa ArrayList.\\

Una explicación para estos resultados es que la implementación personalizada de LinkedList utiliza búsqueda secuencial, lo que causa que al aumentar la cantidad de puntos se requiera una cantidad considerable de iteraciones dentro de la lista para las operaciones que requieren los algoritmos, causando el aumento en el tiempo de ejecución promedio.